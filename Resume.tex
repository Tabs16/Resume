% LaTeX resume using res.cls
\documentclass[margin]{res}
\usepackage[colorlinks = true,
            linkcolor = blue,
            urlcolor  = blue,
            citecolor = blue,
            anchorcolor = blue]{hyperref}
\setlength{\textwidth}{5in} % set width of text portion
% \vspace*{-6mm}

\begin{document}
\vspace*{-2.0cm}
% Center the name over the entire width of resume:
 \moveleft.5\hoffset\centerline{\large\bf Tayba Wasim}
 \moveleft.5\hoffset\centerline{\url{https://tabs16.github.io}}
 
 \moveleft\hoffset\vbox{\hrule width\resumewidth height 1pt}\smallskip

\begin{resume}

\section{EDUCATION}

\begin{tabular}{|c|c|c|c|}

\hline
\textbf{Degree} & \textbf{Year} & {Institution/Board} & {CPI/\%} \\
\hline
B.Tech & 2019 (Expected) & LNM Institute of Information Technology, Jaipur & 8.85/10 \\
\hline
XII & 2015 & The Millennium School, Lucknow (CBSE) & 94.4\% \\
\hline
X & 2013 & The Millennium School, Lucknow (CBSE) & 10 CGPA \\
\hline
\end{tabular}

\section{PROJECTS}

  {\textbf{Data Visualisation}} \href{https://github.com/anantnema/CSI-Project/tree/tayba}{Github} \hfill May `16- July`16\\
  Data Visualisation Project using Javascript under Computer Society of India.

  {\textbf{Wallet - Web Based Application}} \href{https://github.com/Tabs16/qr_crypto-1}{Github} \hfill Nov`16\\
  The Application was developed as a part of Two-day Hackathon. It was made to make money transactions easier.

  {\textbf{MedQA- Medical Application}} \href{https://github.com/MedQA/medqa/tree/tayba}{Github} \hfill May`16 - July`16\\
  It is an application which solves medical queries of people.

  {\textbf{Assembler Design}} \href{https://github.com/Tabs16/ComputerArchitecture-SmartAC}{Github} \hfill March`17\\
  Designed an assembler as part of Computer Organisation and Architecture for a computing machine, that is, Smart Air Conditioner.

  {\textbf{Social Networking Site}} \href{https://github.com/ShubhamSetia/SocialNetworkingSite}{Github} \hfill (In Progress)\\
  Developing a social networking site using Java and MySQL for database management.
  


  \textbf{More projects on \href{https://github.com/Tabs16}{Github} account.}

\section{OPEN \\ SOURCE CONTRIBUTIONS}
	{\textbf{SUGAR LABS}} \hfill December`16 - Present\\
 	Active contributor to Sugar Labs Organisation, specifically to Music Blocks which is a collection of manipulative tools for exploring fundamental musical concepts in an integrative and fun way.

 	{\textbf{MOZILLA}} \hfill August`16 - Present\\
 	Contributed to "Drive nightly reboot for Mozilla" project which involved bug triaging and checking bugs' validity.
 	




\section{TECHNICAL \\ SKILLS}

  {\textbf{Languages}:} C, Python, Java\\
  {\textbf{Web Development}:} JavaScript, HTML/CSS\\
  {\textbf{Frameworks}:} Node.Js(basic), Flask\\
  {\textbf{Databases}:} MySQL, IBM Db2\\
  {\textbf{Libraries Explored}:} Bootstrap, webL10n\\
  {\textbf{Others}:} Latex, Git\\
  {\textbf{Platforms}:} GNU/Linux, MacOSX\\
  {\textbf{Others}:} Inkscape\\

\section{POSITION OF RESPONSIBILITY }
	\begin{itemize}
		\item Vice Chairperson of Computer Society of India ,LNMIIT Chapter.	
  		\item Core team member of Desportivos organising team (Sports fest).
	\end{itemize}

\section{ACHIEVEMENTS}
	\begin{itemize}
		\item Outreachy`17 and Google Summer of Code(GSOC)`17 participant with Sugar Labs (in progress). 
		\item School topper for senior-secondary(2015) .
	\end{itemize}

\section{WORKSHOPS AND CONFERENCES}
	
	\begin{itemize}
		\item Attended conf.KDE.in Conference held in India.
		\item Conducted one-day Linux workshop for college students.
		\item Conducting various workshops for CSI, LNMIIT chapter.
\end{itemize} 

\section{BLOG}
\url{https://tabs16blog.wordpress.com/}

 
\begin{center}
  \begin{footnotesize}
    Last updated: \today \\
  \end{footnotesize}
\end{center}

\end{resume}
\end{document}