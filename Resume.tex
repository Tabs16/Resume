% LaTeX resume using res.cls
\documentclass[margin]{res}
\usepackage[colorlinks = true,
            linkcolor = blue,
            urlcolor  = blue,
            citecolor = blue,
            anchorcolor = blue]{hyperref}
\setlength{\textwidth}{5in} % set width of text portion
% \vspace*{-6mm}

\begin{document}
\vspace*{-2.0cm}
% Center the name over the entire width of resume:
 \moveleft.5\hoffset\centerline{\large\bf Tayba Wasim}
 \moveleft.5\hoffset\centerline{\url{https://tabs16.github.io}}
 
 \moveleft\hoffset\vbox{\hrule width\resumewidth height 1pt}\smallskip

\begin{resume}

\section{EDUCATION}

\begin{tabular}{|c|c|c|c|}

\hline
\textbf{Degree} & \textbf{Year} & {Institution/Board} & {CPI/\%} \\
\hline
B.Tech & 2019 (Expected) & LNM Institute of Information Technology, Jaipur & 8.85/10 \\
\hline
XII & 2015 & The Millennium School, Lucknow (CBSE) & 94.2\% \\
\hline
X & 2013 & The Millennium School, Lucknow (CBSE) & 10 CGPA \\
\hline
\end{tabular}

\section{PROJECTS}

  {\textbf{Data Visualisation}} \href{https://github.com/anantnema/CSI-Project/tree/tayba}{Github} \hfill May `16- July`16\\
  Data Visualisation Project using Javascript under Computer Society of India.

  {\textbf{Wallet - Web Based Application}} \href{https://github.com/Tabs16/qr_crypto-1}{Github} \hfill Nov'16\\
  The Application was developed as a part of Two-day Hackathon. It was made to make money transactions easier.

  {\textbf{MedQA- Medical Application}} \href{https://github.com/MedQA/medqa/tree/tayba}{Github} \hfill May'16 - July'16\\
  It is an application which solves medical queries of people.


  \textbf{More projects on \href{https://github.com/Tabs16}{Github} account.}

 
\begin{center}
  \begin{footnotesize}
    Last updated: \today \\
  \end{footnotesize}
\end{center}

\end{resume}
\end{document}